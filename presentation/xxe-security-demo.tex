%----------------------------------------------------------------------------------------
%	PACKAGES AND THEMES
%----------------------------------------------------------------------------------------
\documentclass[aspectratio=169,xcolor=dvipsnames]{beamer}
\usetheme{Simple}

\usepackage{hyperref}
\usepackage{graphicx}
\usepackage{booktabs}

%----------------------------------------------------------------------------------------
%	TITLE PAGE
%----------------------------------------------------------------------------------------

\title[XXE Security Demo]{XML External Entity (XXE) Vulnerability}
\subtitle{Demonstration, Exploitation \& Mitigation}

\author[Wieczorek, Chwalczyk, Hedrzak, Pytka, Pacyniak]{
    Karol Wieczorek \and 
    Paweł Chwalczyk \and 
    Jakub Hedrzak \and 
    Sebastian Pytka \and 
    Adrian Pacyniak
}
\institute[SUT]{
    Department of Applied Mathematics\\
    Silesian University of Technology
    \vskip 3pt
}
\date{\today}

%----------------------------------------------------------------------------------------
%	PRESENTATION SLIDES
%----------------------------------------------------------------------------------------

\begin{document}

%------------------------------------------------
% TITLE
%------------------------------------------------

\begin{frame}
    \titlepage
\end{frame}

%------------------------------------------------
% OVERVIEW
%------------------------------------------------

\begin{frame}{Overview}
    \textbf{Presentation Structure:}
    \begin{enumerate}
        \item Introduction - Project overview and technologies
        \item Understanding XXE - What it is and how it works
        \item Demonstration - Live attacks and exploits
        \item Real-World Impact - Facebook, Google, Microsoft incidents
        \item Secure Implementation - How to prevent XXE
        \item Results \& Conclusions - Key findings and takeaways
    \end{enumerate}
\end{frame}

%------------------------------------------------
\section{Introduction}
%------------------------------------------------

\begin{frame}{Project Overview}
    \begin{block}{Educational Security Demonstration}
        Comprehensive study of XML External Entity (XXE) vulnerabilities through practical implementation and testing
    \end{block}
    
    \vspace{1em}
    
    \begin{columns}[c]
        \column{.5\textwidth}
        \textbf{Components:}
        \begin{itemize}
            \item Vulnerable web application
            \item Secure implementation
            \item Automated exploit tools
            \item Comprehensive documentation
        \end{itemize}
        
        \column{.5\textwidth}
        \textbf{Technologies:}
        \begin{itemize}
            \item Python 3.8+ / Flask
            \item lxml XML parser
            \item Security testing tools
            \item GitHub repository
        \end{itemize}
    \end{columns}
\end{frame}

%------------------------------------------------
\section{Understanding XXE}
%------------------------------------------------

\begin{frame}{What is XXE?}
    \begin{alertblock}{XML External Entity (XXE) Injection}
        A web security vulnerability that allows attackers to interfere with XML data processing, enabling:
    \end{alertblock}
    
    \vspace{1em}
    
    \begin{columns}[c]
        \column{.33\textwidth}
        \textbf{File Disclosure}
        \begin{itemize}
            \item Read local files
            \item Access sensitive data
            \item Retrieve credentials
        \end{itemize}
        
        \column{.33\textwidth}
        \textbf{SSRF}
        \begin{itemize}
            \item Internal requests
            \item Port scanning
            \item Network mapping
        \end{itemize}
        
        \column{.33\textwidth}
        \textbf{Denial of Service}
        \begin{itemize}
            \item Billion Laughs
            \item Resource exhaustion
            \item Server crash
        \end{itemize}
    \end{columns}
\end{frame}

%------------------------------------------------

\begin{frame}[fragile]{How XXE Works}
    \begin{block}{XML External Entities}
        XML allows defining custom entities that can reference external resources
    \end{block}
    
    \vspace{0.5em}
    
    \textbf{Normal XML:}
    \begin{verbatim}
<?xml version="1.0"?>
<user>
  <name>John Doe</name>
  <email>john@example.com</email>
</user>
    \end{verbatim}
\end{frame}

%------------------------------------------------

\begin{frame}[fragile]{XXE Payload Example}
    \textbf{Malicious XXE Payload:}
    \begin{verbatim}
<?xml version="1.0"?>
<!DOCTYPE foo [
  <!ENTITY xxe SYSTEM "file:///etc/passwd">
]>
<data>&xxe;</data>
    \end{verbatim}
    
    \vspace{1em}
    
    \begin{alertblock}{Impact}
        This payload reads the system's password file and returns it to the attacker
    \end{alertblock}
\end{frame}

%------------------------------------------------

\begin{frame}{XXE Attack Flow}
    \begin{enumerate}
        \item \textbf{Attacker} sends malicious XML with external entity
        \item \textbf{Application} parses XML with vulnerable parser
        \item \textbf{Parser} resolves external entity reference
        \item \textbf{Parser} reads file/makes request
        \item \textbf{Application} returns data to attacker
    \end{enumerate}
    
    \vspace{1em}
    
    \begin{alertblock}{Critical Requirement}
        XML parser must have \texttt{resolve\_entities=True} (VULNERABLE configuration)
    \end{alertblock}
\end{frame}

%------------------------------------------------
\section{Demonstration}
%------------------------------------------------

\begin{frame}{Vulnerable Application}
    \begin{figure}
        \includegraphics[width=0.95\textwidth]{../docs/screenshots/01-vulnerable-homepage.png}
        \caption{Intentionally vulnerable Flask application with XXE parser}
    \end{figure}
\end{frame}

%------------------------------------------------

\begin{frame}{Attack 1: File Disclosure - System Files}
    \textbf{Target:} \texttt{/etc/passwd}
    
    \begin{figure}
        \includegraphics[width=0.8\textwidth]{../docs/screenshots/03-xxe-file-disclosure-passwd.png}
        \caption{Successful file disclosure: 9,344 bytes of system user data exposed}
    \end{figure}
\end{frame}

%------------------------------------------------

\begin{frame}{Attack 2: File Disclosure - Application Secrets}
    \textbf{Target:} \texttt{sensitive\_data.txt}
    
    \begin{figure}
        \includegraphics[width=0.8\textwidth]{../docs/screenshots/04-xxe-sensitive-data.png}
        \caption{Critical breach: Database credentials, API keys, and internal IPs exposed (456 bytes)}
    \end{figure}
\end{frame}

%------------------------------------------------

\begin{frame}{Automated Exploitation}
    \textbf{Python Exploit Script:} File Disclosure
    
    \begin{figure}
        \includegraphics[width=0.9\textwidth]{../docs/screenshots/08-exploit-file-disclosure.png}
        \caption{Automated exploit successfully retrieved /etc/passwd}
    \end{figure}
\end{frame}

%------------------------------------------------

\begin{frame}{Attack 3: SSRF - Port Scanning}
    \textbf{Technique:} Server-Side Request Forgery via XXE
    
    \begin{figure}
        \includegraphics[width=0.9\textwidth]{../docs/screenshots/09-exploit-ssrf-scan.png}
        \caption{Port scanning through vulnerable XML parser}
    \end{figure}
    
    \begin{alertblock}{Note}
        Modern lxml has security features that limit SSRF effectiveness
    \end{alertblock}
\end{frame}

%------------------------------------------------

\begin{frame}{Attack 4: Denial of Service}
    \textbf{Technique:} Billion Laughs / XML Bomb
    
    \begin{figure}
        \includegraphics[width=0.9\textwidth]{../docs/screenshots/10-exploit-dos-blocked.png}
        \caption{DoS attempt blocked by lxml's entity expansion limits}
    \end{figure}
\end{frame}

%------------------------------------------------
\section{Real-World Impact}
%------------------------------------------------

\begin{frame}{Real-World XXE Incidents}
    \begin{block}{Facebook (2013) - \$33,500 Bug Bounty}
        \begin{itemize}
            \item XXE in OpenID authentication handler
            \item File disclosure vulnerability
            \item Escalated to Remote Code Execution
            \item Researcher: Reginaldo Silva
        \end{itemize}
    \end{block}
    
    \begin{block}{Google (2012)}
        \begin{itemize}
            \item XXE in AppEngine and Blogger
            \item Read-only access to production servers
            \item Same researcher as Facebook incident
        \end{itemize}
    \end{block}
\end{frame}

%------------------------------------------------

\begin{frame}{Real-World XXE Incidents (Continued)}
    \begin{block}{Android Development Tools (2017) - "ParseDroid"}
        \begin{itemize}
            \item APKTool, Android Studio, Eclipse, IntelliJ IDEA affected
            \item XXE in DocumentBuilderFactory XML parser
            \item Source code theft, supply chain attacks possible
            \item Discovered by Check Point Research
        \end{itemize}
    \end{block}
    
    \begin{block}{Microsoft SharePoint (CVE-2019-0604)}
        \begin{itemize}
            \item Critical RCE via XXE
            \item Exploited by APT group Emissary Panda
            \item Active exploitation for 9+ months after patch
            \item CVSS Score: 9.8 (Critical)
        \end{itemize}
    \end{block}
\end{frame}

%------------------------------------------------

\begin{frame}{Impact Statistics}
    \begin{table}
        \begin{tabular}{l l l}
            \toprule
            \textbf{Company} & \textbf{Year} & \textbf{Impact} \\
            \midrule
            Facebook & 2013 & RCE, \$33.5k bounty \\
            Google & 2012 & Server file access \\
            Android Tools & 2017 & Millions of developers \\
            SharePoint & 2019 & APT exploitation \\
            \bottomrule
        \end{tabular}
        \caption{Notable XXE vulnerabilities in major platforms}
    \end{table}
    
    \vspace{1em}
    
    \begin{alertblock}{Key Finding}
        XXE vulnerabilities affect even the most security-conscious organizations
    \end{alertblock}
\end{frame}

%------------------------------------------------
\section{Secure Implementation}
%------------------------------------------------

\begin{frame}{Secure Application}
    \begin{figure}
        \includegraphics[width=0.95\textwidth]{../docs/screenshots/06-secure-homepage.png}
        \caption{Secure implementation with XXE protection enabled}
    \end{figure}
\end{frame}

%------------------------------------------------

\begin{frame}{XXE Attack Blocked}
    \textbf{Same payload, different result:}
    
    \begin{figure}
        \includegraphics[width=0.8\textwidth]{../docs/screenshots/07-secure-xxe-blocked.png}
        \caption{External entity not resolved - attack prevented}
    \end{figure}
\end{frame}

%------------------------------------------------

\begin{frame}[fragile]{Vulnerable Configuration}
    \textbf{DANGEROUS - External entities enabled:}
    
    \begin{verbatim}
parser = etree.XMLParser(
    resolve_entities=True,   # XXE vulnerability!
    no_network=False,        # SSRF possible
    load_dtd=True,          # Entity expansion
    huge_tree=True          # No DoS limits
)
    \end{verbatim}
    
    \begin{alertblock}{Problems}
        \begin{itemize}
            \item External entities processed
            \item Network access allowed
            \item DTD loading enabled
        \end{itemize}
    \end{alertblock}
\end{frame}

%------------------------------------------------

\begin{frame}[fragile]{Secure Configuration}
    \textbf{SAFE - External entities disabled:}
    
    \begin{verbatim}
parser = etree.XMLParser(
    resolve_entities=False,  # XXE prevented!
    no_network=True,         # SSRF blocked
    load_dtd=False,         # No expansion
    huge_tree=False         # DoS protection
)
    \end{verbatim}
    
    \begin{block}{Protection}
        \begin{itemize}
            \item External entities disabled
            \item Network access blocked
            \item DTD loading disabled
        \end{itemize}
    \end{block}
\end{frame}

%------------------------------------------------

\begin{frame}{Mitigation Best Practices}
    \begin{enumerate}
        \item \textbf{Disable External Entities}
        \begin{itemize}
            \item Primary defense: \texttt{resolve\_entities=False}
            \item Prevents all XXE attacks
        \end{itemize}
        
        \item \textbf{Block Network Access}
        \begin{itemize}
            \item \texttt{no\_network=True}
            \item Prevents SSRF attacks
        \end{itemize}
        
        \item \textbf{Disable DTD Loading}
        \begin{itemize}
            \item \texttt{load\_dtd=False}
            \item Prevents entity expansion
        \end{itemize}
        
        \item \textbf{Input Validation}
        \begin{itemize}
            \item Validate XML structure
            \item Limit file size
            \item Sanitize user input
        \end{itemize}
    \end{enumerate}
\end{frame}

%------------------------------------------------

\begin{frame}{Defense in Depth}
    \begin{columns}[c]
        \column{.33\textwidth}
        \textbf{Layer 1: Parser}
        \begin{itemize}
            \item Secure configuration
            \item Disable entities
            \item Block network
        \end{itemize}
        
        \column{.33\textwidth}
        \textbf{Layer 2: Input}
        \begin{itemize}
            \item Validate format
            \item Size limits
            \item Type checking
        \end{itemize}
        
        \column{.33\textwidth}
        \textbf{Layer 3: System}
        \begin{itemize}
            \item Least privilege
            \item File permissions
            \item Network isolation
        \end{itemize}
    \end{columns}
    
    \vspace{2em}
    
    \begin{block}{Key Principle}
        Multiple security layers provide better protection than a single control
    \end{block}
\end{frame}

%------------------------------------------------
\section{Results \& Conclusions}
%------------------------------------------------

\begin{frame}{Testing Results Summary}
    \begin{table}
        \begin{tabular}{l c c l}
            \toprule
            \textbf{Attack Type} & \textbf{Vulnerable} & \textbf{Secure} & \textbf{Impact} \\
            \midrule
            File Disclosure & FAIL & PASS & Critical \\
            Sensitive Data & FAIL & PASS & Critical \\
            SSRF & PARTIAL & PASS & Medium \\
            DoS & PASS & PASS & Low \\
            \bottomrule
        \end{tabular}
        \caption{Security testing results comparison}
    \end{table}
    
    \vspace{1em}
    
    \begin{block}{Key Finding}
        Single configuration change (\texttt{resolve\_entities=False}) prevents critical XXE attacks
    \end{block}
\end{frame}

%------------------------------------------------

\begin{frame}{Project Deliverables}
    \begin{columns}[c]
        \column{.5\textwidth}
        \textbf{Implemented:}
        \begin{itemize}
            \item Vulnerable Flask application
            \item Secure Flask application
            \item 3 Python exploit scripts
            \item Comprehensive documentation
            \item 10 demonstration screenshots
            \item GitHub repository
        \end{itemize}
        
        \column{.5\textwidth}
        \textbf{Demonstrated:}
        \begin{itemize}
            \item File disclosure attacks
            \item SSRF techniques
            \item DoS attempts
            \item Automated exploitation
            \item Secure configuration
            \item Real-world context
        \end{itemize}
    \end{columns}
    
    \vspace{1em}
    
    \begin{block}{Repository}
        \url{https://github.com/Fablek/xxe-security-demo}
    \end{block}
\end{frame}

%------------------------------------------------

\begin{frame}{Key Takeaways}
    \begin{enumerate}
        \item \textbf{XXE is Critical} - Can lead to complete system compromise
        \item \textbf{Simple to Exploit} - Requires only XML input capability
        \item \textbf{Affects Major Companies} - Facebook, Google, Microsoft all vulnerable
        \item \textbf{Easy to Fix} - One configuration change prevents attacks
        \item \textbf{Testing Essential} - Both manual and automated testing needed
        \item \textbf{Defense in Depth} - Multiple security layers recommended
    \end{enumerate}
    
    \vspace{1em}
    
    \begin{alertblock}{Critical Message}
        Always review and secure XML parser configurations in production applications
    \end{alertblock}
\end{frame}

%------------------------------------------------

\begin{frame}{References}
    \footnotesize{
        \begin{thebibliography}{99}
            \bibitem{owasp} OWASP Foundation
            \newblock XML External Entity (XXE) Processing
            \newblock \url{https://owasp.org/www-community/vulnerabilities/XML_External_Entity_(XXE)_Processing}
            
            \bibitem{portswigger} PortSwigger
            \newblock XML external entity (XXE) injection
            \newblock \url{https://portswigger.net/web-security/xxe}
            
            \bibitem{facebook} Reginaldo Silva (2013)
            \newblock How I found a Remote Code Execution bug affecting Facebook's servers
            \newblock \url{https://www.ubercomp.com/posts/2014-01-16_facebook_remote_code_execution}
            
            \bibitem{checkpoint} Check Point Research (2017)
            \newblock ParseDroid: Targeting The Android Development \& Research Community
            \newblock \url{https://research.checkpoint.com/2017/parsedroid-targeting-android-development-research-community/}
        \end{thebibliography}
    }
\end{frame}

%------------------------------------------------

\begin{frame}
    \Huge{\centerline{Thank You}}
    \vspace{1em}
    \large{\centerline{Questions?}}
    \vspace{2em}
    \normalsize
    \begin{center}
        \textbf{Authors:}\\
        Karol Wieczorek, Paweł Chwalczyk, Jakub Hedrzak,\\
        Sebastian Pytka, Adrian Pacyniak\\
        \vspace{0.5em}
        GitHub: \url{https://github.com/Fablek/xxe-security-demo}\\
        \vspace{0.5em}
        \textit{Educational Project - Use Responsibly}
    \end{center}
\end{frame}

%----------------------------------------------------------------------------------------

\end{document}